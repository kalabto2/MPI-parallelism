Jako semestrální práci jsem měl zadanou úlohu \textbf{MBP: bipartitní podgraf s maximální váhou}.
Úkolem je nalézt podmnožinu hran \textbf{F} takovou, že podgraf \textbf{G}(\textbf{V},\textbf{F}) je souvislý a bipartitní a váha \textbf{F} je maximální v rámci všech možných bipartitních souvislých podgrafů \textbf{G} nad \textbf{V}. Přičemž bipartitní graf je ten, jemuž lze množinu uzlů \textbf{V} rozdělit na disjunktní podmnožiny \textbf{U} a \textbf{W} tak, že každá hrana v \textbf{E} spojuje uzel z \textbf{U} s uzlem z \textbf{W}. 
Tato úloha byla omezena na takové grafy, jež mají váhy hran z intervalu \( <80, 120> \).

Tato úloha byla řešena sekvenčně následovně. Nejprve jsem označil všechny vrcholy grafu \textit{nedefinovaný} status. První vrchol jsem označil jako \textit{část 1}. Následně jsem pomocí stromové rekurze přidával nebo nepřidával hrany a označoval vrcholy jako \textit{část 1} nebo \textit{část 2}.
Pro co největší urychlení výpočtu jsem vytvářel takové podgrafy, které bipartitní můžou být. Tedy pokud jsem přidal hranu která již sousedila s částí 2, druhý vrchol jsem nutně obarvil částí 1 a nezkoušel jinou možnost.
Druhým zrychlením bylo, že jsem hranu která měla oba vrcholy již validně obarvené, přidal a nezkoušel ji nepřidat.
Třetím zrychlením bylo přidání metody ořezávaní \textit{branch and bound}. Ta spočívala v principu zapamatování si dosavadní nejlepší konfigurace podgrafů. Pokud se při výpočtu narazí na stav, který již nemá šanci zlepšit dosavadní maximum, pak se tento podstrom stavů přeskočí.

Dalším případným zrychlením může být seřazení hran v sestupném pořadí podle váhy, aby maximální konfigurace byla nalezena co nejdříve a ořezávání bylo tak co nejefektivnější. To už ale má implementace nezahrnuje.

Implementace počítá všechny instance, které jsou v nad-složce \verb|graf_mbp|
přičemž každá instance musí být popsána souborem 

\verb|graf_<počet vrcholů grafu>_<průměrný stupeň uzlu grafu>.txt|.

Výstupem je pak maximální váha, počet těchto řešení a informační údaje jako čas výpočtu či počet rekurzí.

Pro demonstraci výsledků jsem vygeneroval instance pro výpočet. 
Jejich čas výpočtu je zachycen v následující tabulce.

\FloatBarrier
\begin{table}[]
    \begin{tabular}{l|rrr}
    {}                  instance &    čas &  počet vláken &  počet procesů \\
    \hline
    graf\_15\_8.txt &  116.05 &           1 &         1 \\
    graf\_18\_7.txt &  227.07 &           1 &         1 \\
    graf\_21\_6.txt &  145.42 &           1 &         1 \\
    \end{tabular}
\end{table}
\FloatBarrier
